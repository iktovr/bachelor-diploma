\introduction

Актуальность темы данной работы связана с распространенностью нелинейных динамических систем для описания различных процессов. В случаях, когда существуют только дискретные данные, собранные, например, в результате эксперимента, возникает задача по идентификации системы, лежащей за этими данными. Такая задача может осложняться большим объемом данных, их сложностью или наличием в них посторонних шумов, что делает проблематичным ее решение в ручном режиме.

Таким образом, выполненная работа актуальна и с теоретической, и с практической точек зрения.

Цель работы --– идентификация систем обыкновенных дифференциальных уравнений первого порядка на основе потенциально шумных данных.

Для достижения поставленной цели в работе были решены следующие задачи:
\begin{itemize}
  \item реализация алгоритма идентификации;
  \item реализация алгоритмов дифференцирования шумных данных;
  \item реализация алгоритмов разреженной регрессии;
  \item тестирование алгоритма идентификации на известных системах ОДУ 1-ого порядка;
  \item сравнение различных методов дифференцирования и регрессии.
\end{itemize}

Работа основывалась на следующих инструментах и методах:
\begin{itemize}
  \item язык программирования Python,
  \item библиотеки научных вычислений NumPy и SciPy,
  \item библиотека машинного обучения Scikit-learn,
  \item библиотеки построения графиков Matplotlib и Seaborn,
  \item среда разработки Jupyter.
\end{itemize}

Основными результатами, полученными в работе, являются:
\begin{itemize}
\item работающий алгоритм идентификации систем ОДУ по шумным данным;
\item рекомендации по использованию алгоритма идентификации;
\item реализация алгоритмов численного дифференцирования методом регуляризации полной вариации;
\item рекомендации по подбору параметров и использованию алгоритмов дифференцирования.
\end{itemize}

Результаты работы предназначены для использования в анализе данных.

Использование разработки позволяет выявлять в данных закономерности, которые могут быть описаны при помощи нелинейных динамических систем различного вида, например, систем ОДУ 1-ого порядка. Это существенно расширяет возможности анализа данных, так как позволяет использовать соответствующий математический аппарат для анализа самой системы.
