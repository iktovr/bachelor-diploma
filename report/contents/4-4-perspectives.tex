\subsection{Перспективы дальнейшего развития алгоритма}

В данной работе задача идентификации ограничивается только системами ОДУ 1-ого порядка. Однако возможности алгоритма позволяют делать гораздо больше этого.

Рассмотрим описанный базовый алгоритм. Имея методы обычного численного дифференцирования первого порядка, можно довольно легко получить методы дифференцирования высоких порядков, равно как и методы получения частных производных. Это позволяет производить идентификацию дифференциальных уравнений в частных производных \cite{sindy}, что довольно сильно расширяет область задач, к которым применим данный алгоритм. Также можно отказаться от производных вообще и идентифицировать системы рекуррентных соотношений \cite{sindy} вида:

\begin{equation}
\textsc{x}_{k+1} = f(\textsc{x}_k).
\end{equation}

Кроме этого, можно расширять сам алгоритм идентификации для решения других классов задач. Например, задач теории управления, в частности управления с прогнозирующими моделями (model predictive control) \cite{sindyc_1, sindyc_2}. Такая модификация позволяет идентифицировать системы, в случае ОДУ, вида:

\begin{equation}
\dot{\textsc{x}} = f(\textsc{x}, u),
\end{equation}

где $u$ --- управляющий сигнал.

Или, вместо задачи Коши, идентифицировать системы уравнений краевой задачи \cite{sindy-bvp}. Или все ту же задачу Коши, но для уравнений в неявном виде \cite{sindy-pi} (для таких задач процесс разреженной регрессии еще и крайне хорошо распараллеливается):

\begin{equation}
f(\textsc{x}, \dot{\textsc{x}}) = 0.
\end{equation}

Однако также можно улучшать сам алгоритм. Например, использовать идею ансамблей и обучать несколько моделей вместо одной \cite{sindy-ensemble}. При этом каждая из моделей обучается на подмножестве исходных данных (бутстрэппинг) и идентифицированные системы агрегируются в одну итоговую.

Что касается отдельных частей алгоритма, помимо использования специализированных методов дифференцирования можно пытаться фильтровать сами данные, например, при помощи нейросетевого подхода \cite{nn_denoise}, который отделяет шумовую компоненту, что позволяет работать с шумом любого распределения и даже это распределение идентифицировать.
