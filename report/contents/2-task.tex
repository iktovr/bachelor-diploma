\section{ПОСТАНОВКА ЗАДАЧИ}

\subsection{Актуальность работы}

Извлечение математических закономерностей из данных --- структурный анализ --- является важной задачей во многих научных областях. Существует много проблем, связанных с данными, таких как понимание когнитивных процессов на основе мозговой активности, изучение закономерностей климата, определение устойчивости финансовых рынков, прогнозирование и подавление распространения болезней. В связи с обилием данных, роль структурного анализа в этих сферах вероятно будет только расти.

Достижения в области машинного обучения и науке о данных привели к продвижению в анализе и понимании сложных данных, извлечение закономерностей из которых превышает возможности человека. Однако, несмотря на быстрое развитие инструментов на основе статистических отношений, замедлился прогресс в извлечении физических моделей динамических процессов. Это приводит к тому, что имеющиеся методы не могут экстраполировать результат за пределы аттрактора, в котором данные были получены.

Поэтому более перспективными кажутся методы, извлекающие закономерности в символьном виде. Первым прорывом в этой области стало использование символьной регрессии и генетического программирования для поиска нелинейных дифференциальных уравнений \cite{symbolic, genetic}. В таком подходе балансируется сложность модели, измеряемая количеством членов, с точностью идентификации. Однако символьная регрессия вычислительно сложна, а также плохо масштабируется и склонна к переобучению.

Другим методом, который и рассматривается в данной работе, является решение задачи идентификации с точки зрения разреженной регрессии \cite{sindy, lasso}. При этом активно используется тот факт, что большинство физических систем имеют небольшое количество значимых членов, что делает их разреженными в пространстве нелинейных функций. Это позволяет избежать комбинаторного взрыва, который возникает при обычном переборе. В результате процесс идентификации обеспечивает естественное балансирование сложности модели, которая определяется разреженностью правых частей уравнений, с точностью. Использование алгоритмов выпуклой оптимизации позволяет применять метод к задачам большого масштаба.

В таком подходе также можно усмотреть сходство с методом разложения по динамическим модам (dynamic mode decomposition) \cite{dmd}, который является линейной динамической регрессией. Этот метод опирается только на исходные данные, а не на знание уравнений динамики, и также связан с теорией операторов Купмана \cite{koopman}. Однако при использовании этого метода все равно необходимы предположения о виде динамической системы, поскольку на данный момент нет методов по определению наблюдаемых функций. В отличие от этого, использование разреженной регрессии позволяет автоматически определять значимые члены в динамических системах.

\subsection{Техническое задание}

Таким образом, целью работы является реализация алгоритма идентификации нелинейных систем на основе данных.

Входные данные представляют из себя массив замеров некоторых величин, описывающих динамическую систему. Так как алгоритм нацелен на использование с реальными системами и данными, полученными экспериментальным путем, входные данные могут содержать некоторую шумовую компоненту. Поэтому разработанный алгоритм должен быть устойчив к этому шуму.

Задача идентификация ограничивается системами обыкновенных дифференциальных уравнений первого порядка. Системы ОДУ 1-ого порядка позволяют описать довольно много процессов реального мира и для их идентификации достаточно базовой версии алгоритма и производных первого порядка.

В качестве источника данных используются известные системы ОДУ. Данные получаются синтетическим путем.
