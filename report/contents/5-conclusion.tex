\conclusion

В ходе выполнения выпускной квалификационной работы бакалавра был реализован алгоритм идентификации систем ОДУ на основе данных. Было произведено тестирование алгоритма на известных системах, по результатам которого можно заключить о его работоспособности. Отдельно от алгоритма были также протестированы и проанализированы его составные части: алгоритмы дифференцирования и разреженной регрессии.

Большую часть работы заняло численное дифференцирование. Применительно к задаче идентификации методы регуляризации полной вариации ранее не применялись, поэтому реализация, анализ, подбор параметров осуществлялись полностью с нуля. С этой точки зрения, использование этих методов можно считать некоторым развитием данной области.

В ходе анализа методов дифференцирования была обнаружена зависимость между уровнем шума и параметром метода. Поэтому для эффективного противодействия шуму в данных, необходимо оценивать его уровень. Этот вопрос не рассматривался в данной работе, однако, для этого существуют различные методы, например, с использованием спектрального анализа.

Алгоритм разреженной регрессии Lasso, сначала показался вполне работоспособным, однако, при тестировании для задачи идентификации работоспособность не подтвердилась. Таким образом, этот алгоритм отброшен.

Тестирование алгоритма идентификации показало сложность данной задачи, отдельно подсветив тот факт, что алгоритмы регрессии и дифференцирования, хорошо работающие в своей области, не достаточно просто объединить, и такие проблемы, как подбор параметров, приходится решать снова, учитывая специфику задачи. Однако с другой стороны, полученные результаты вселяют уверенность в перспективность данного метода и возможность его успешного использования в практических задачах.